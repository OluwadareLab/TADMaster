\nonstopmode{}
\documentclass[letterpaper]{book}
\usepackage[times,inconsolata,hyper]{Rd}
\usepackage{makeidx}
\usepackage[utf8,latin1]{inputenc}
% \usepackage{graphicx} % @USE GRAPHICX@
\makeindex{}
\begin{document}
\chapter*{}
\begin{center}
{\textbf{\huge Package `CaTCH'}}
\par\bigskip{\large \today}
\end{center}
\begin{description}
\raggedright{}
\item[Type]\AsIs{Package}
\item[Title]\AsIs{Call a hierarchy of domains based on Hi-C data}
\item[Version]\AsIs{1.0}
\item[Date]\AsIs{2016-07-05}
\item[Author]\AsIs{Yinxiu Zhan}
\item[Maintainer]\AsIs{Yinxiu Zhan }\email{yinxiu.zhan@fmi.ch}\AsIs{}
\item[Imports]\AsIs{parallel}
\item[Description]\AsIs{This package allows building the hierarchy of
domains starting from Hi-C data. Each hierarchical
level is identified by a minimum value of physical
insulation between neighboring domains.}
\item[License]\AsIs{GPL-2 or later}
\item[NeedsCompilation]\AsIs{yes}
\end{description}
\Rdcontents{\R{} topics documented:}
\inputencoding{utf8}
\HeaderA{CaTCH-package}{Call a hierarchy of domains based on Hi-C data}{CaTCH.Rdash.package}
\aliasA{CaTCH}{CaTCH-package}{CaTCH}
\keyword{TADs, domain calling, Hi-C}{CaTCH-package}
%
\begin{Description}\relax
This package allows building the hierarchy of
domains starting from Hi-C data. Each hierarchical
level is identified by a minimum value of physical
insulation between neighboring domains.
\end{Description}
%
\begin{Details}\relax

The DESCRIPTION file:

\Tabular{ll}{
Package: & CaTCH\\{}
Type: & Package\\{}
Title: & Call a hierarchy of domains based on Hi-C data\\{}
Version: & 1.0\\{}
Date: & 2016-07-05\\{}
Author: & Yinxiu Zhan\\{}
Maintainer: & Yinxiu Zhan <yinxiu.zhan@fmi.ch>\\{}
Imports: & parallel\\{}
Description: & This package allows building the hierarchy of
domains starting from Hi-C data. Each hierarchical
level is identified by a minimum value of physical
insulation between neighboring domains.\\{}
License: & GPL-2 or later\\{}
NeedsCompilation: & yes\\{}
Packaged: & 2018-07-10 15:18:41 UTC; yinxiu\\{}
}

Index of help topics:
\begin{alltt}
CaTCH-package           Call a hierarchy of domains based on Hi-C data
domain.call             Call a hierarchy of domains based on Hi-C data
\end{alltt}

\end{Details}
%
\begin{Author}\relax
Yinxiu Zhan

Maintainer: Yinxiu Zhan <yinxiu.zhan@fmi.ch>
\end{Author}
%
\begin{References}\relax
Zhan et al, Reciprocal insulation analysis of Hi-C data shows that TADs represent a functionally but not structurally privileged scale in the hierarchical folding of chromosomes, Genome Research 2017
\end{References}
\inputencoding{utf8}
\HeaderA{domain.call}{Call a hierarchy of domains based on Hi-C data}{domain.call}
\methaliasA{domain.call.parallel}{domain.call}{domain.call.parallel}
\keyword{TADs}{domain.call}
\keyword{hierarchy}{domain.call}
%
\begin{Description}\relax
This package allows building the hierarchy of domains
starting from Hi-C data. Each hierarchical level is identified
by a minimum value of physical insulation between
neighboring domains.
\end{Description}
%
\begin{Usage}
\begin{verbatim}
domain.call(input)
domain.call.parallel(inputs,ncpu=parallel::detectCores()-1L)
\end{verbatim}
\end{Usage}
%
\begin{Arguments}
\begin{ldescription}
\item[\code{input}] 
File containing Hi-C data for the SINGLE CHROMOSOME with 6 columns:
\begin{itemize}

\item{} col1 = chromosome
\item{} col2 = genomic coordinate of the start region
\item{} col3 = genomic coordinate of the end region
\item{} col4 = bin of the start region (genomic coordinate divided by binsize)
\item{} col5 = bin of the end region
\item{} col6 = Hi-C counts

\end{itemize}

************OR*************

File containing Hi-C data for the SINGLE CHROMOSOME with 4 columns:
\begin{itemize}

\item{} col1 = chromosome
\item{} col2 = bin of the start region (genomic coordinate divided by binsize)
\item{} col3 = bin of the end region (genomic coordinate divided by binsize)
\item{} col4 = Hi-C counts

\end{itemize}


\item[\code{inputs}] 
A character vector of files as \code{input}

\item[\code{ncpu}] 
Number of cpu that you want to use (Default ncpu=4)


\end{ldescription}
\end{Arguments}
%
\begin{Value}
Output of

\begin{ldescription}
\item[\code{\code{domain.call}}]  A list with two elements
\begin{itemize}

\item{} ncluster: A data.frame with 3 columns

-Chromosome (chromosome)

-Reciprocal insulation (RI)

-Number of domains (ndomains)



\item{} clusters: A data.frame with 5 columns

-Chromosome (chromosome)

-Reciprocal insulation (RI)

-Start of domain (start)

-End of domain (end)

-Actual insulation (insulation)



\end{itemize}
	


\item[\code{\code{domain.call.parallel}}]   A list of outputs as in \code{domain.call} for each file	
\end{ldescription}
\end{Value}
%
\begin{Author}\relax
Yinxiu Zhan
\end{Author}
%
\begin{References}\relax
Zhan et al....
\end{References}
%
\begin{Examples}
\begin{ExampleCode}
#R code to be here
	fileinput=system.file("Test.dat.gz",package="CaTCH")
	library(CaTCH)
	domain.call(fileinput)
\end{ExampleCode}
\end{Examples}
\printindex{}
\end{document}
